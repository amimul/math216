\section{Toward schemes}

\begin{exercise} % TODO do this
  I should be ashamed of myself, not being able to answer this question.
\end{exercise}

\begin{exercise} % TODO do this
  I should be ashamed of myself, not being able to answer this question.
\end{exercise}


\section{The underlying set of affine schemes}

\begin{exercise}
  \label{exercise:42a}
  \begin{enumerate}
    \item\label{enumerate:42a-a} So we're looking for the prime ideals of~$\Spec k[\epsilon]/(\epsilon^2)$. But these correspond to the prime ideals of~$\Spec k[\epsilon]$ containing~$(\epsilon^2)$. Now the only prime ideal of this form is~$(\epsilon)$. This corresponds to the polynomials in~$\epsilon$ with no constant term. If there would be a constant term, \ie, something of the form~$a+b\epsilon$ it would be invertible modulo~$\epsilon^2$ using a geometric series. There is only one point.

      Notice that~$\Spec k[\epsilon]/(\epsilon)$ is not an integral domain:~$\epsilon^2$ is contained in~$(0)$ yet~$\epsilon\notin(0)$.

    \item\label{enumerate:42a-b} By commutative algebra the prime ideals of the localization correspond to the prime ideals of~$\Spec k[x]$ not containing~$(x)$. So the set~$\Spec k[x]_{(x)}$ corresponds to~$\Spec k[x]\setminus\left\{ (x) \right\}$ because there is only one prime ideal containing~$x$ namely~$(x)$: if there would be another one we could reduce it to a constant ending up the whole ring, a contradiction.
  \end{enumerate}
\end{exercise}

\begin{exercise}
  Using the discriminant we obtain the two roots of the quadratic which look like
  \begin{equation}
    x_{1,2}=\frac{-a\pm\sqrt{a^2-4b}}{2}
  \end{equation}
  where~$a^2-4b<0$. Now using operations of~$\mathbb{R}$ we can reduce this to~$i$.
\end{exercise}

\begin{exercise}
  This set corresponds to all polynomials that are irreducible over~$\mathbb{Q}$. There are the obvious points~$(x-a)$ where~$a\in\mathbb{Q}$, but all roots of polynomials are present too but they are glued together by the corresponding Galois actions. It corresponds to the identification of roots in the algebraic closure~$\mathbb{Q}^{\alg}$.
\end{exercise}

\begin{exercise}
  \label{exercise:42d}
  Suppose~$\mathfrak{p}$ is a prime ideal that is not a principal ideal. Take two essential generators~$f(x,y)$ and~$g(x,y)$ (\ie, with not all factors of one contained in the other). This must be possible because otherwise we wouldn't have a principal ideal: one can be written as a product of the other with a polynomial containing the missing factors. Now because~$\mathfrak{p}$ is prime we can remove all common factors.

  By applying the Euclidean algorithm in~$\mathcal{C}(x)[y]$ we can find a polynomial in the variable~$x$ contained in~$(f(x,y),g(x,y))\subseteq\mathfrak{p}$, which by the algebraic closedness of~$\mathcal{C}$ reduces to a linear factor~$(x-a)$ contained in~$\mathfrak{p}$ and analogously~$(y-b)\in\mathfrak{p}$.

  Obviously any principal ideal must be generated by an irreducible polynomial. So having reduced all non-principal ideals to ideals of the form~$(x-a,y-b)$ we have finished the proof.
\end{exercise}

\begin{exercise}
  The first maximal ideal is~$(x^2+y^2-4,x-y)$ while the second is~$(x^2+y^2-4,x+y)$. The residue fields are~$\mathbb{Q}(\sqrt{2})$ in both cases: substituting the second generator in the first yields this result.
\end{exercise}

\begin{exercise} % TODO do this
  I'm still wondering whether my answer is correct.
\end{exercise}

\begin{exercise}
  \label{exercise:42g}
  I have used this fact in~\autoref{exercise:42a}\ref{enumerate:42a-a}. It boils down to
  \begin{equation}
    A/J\cong (A/I)/(J/I) 
  \end{equation}
  where~$I\subseteq J$ are prime ideals of~$A$, and this is equivalent to~$\overline{J}$ being prime in~$A/I$.
\end{exercise}

\begin{exercise} % TODO rewrite this
  \label{exercise:42h}
  I have used this fact in~\autoref{exercise:42a}\ref{enumerate:42a-b}. A prime ideal of~$A$ that contains an element of~$S$ will the whole become~$S^{-1}A$ under localization. A prime ideal of~$A$ disjoint of~$S$ remains prime because the product~$(p_1/s_1)(p_2/s_2)$ is in the prime ideal if and only if~$(p_1p_2)/(s_1s_2)$ is in the prime ideal, but we can multiply with~$s_1s_2$ as this is by multiplicativity of~$S$ not a member of the prime ideal. Now we have reduced it to~$p_1p_2$ in the prime ideal.
\end{exercise}

\begin{exercise} % TODO do this
  I haven't figured it out yet.
\end{exercise}

\begin{exercise}
  \label{exercise:42j}
  Assume~$b_1b_2\in\phi^{-1}(\mathfrak{p})$, we have
  \begin{equation}
    \phi(b_1b_2)=\phi(b_1)\phi(b_2)\in\phi\left( \phi^{-1}(\mathfrak{p}) \right)=\mathfrak{p},
  \end{equation}
  hence~$\phi(b_1)$ or~$\phi(b_2)$ as~$\phi(\phi^{-1}(\mathfrak{p}))=\mathfrak{p}$ is a prime ideal in~$A$. We obtain that~$\phi^{-1}(\phi(b_1))=b_1$ or~$\phi^{-1}(\phi(b_2))=b_2$ must be an element of~$\phi^{-1}(\mathfrak{p})$.
\end{exercise}

\begin{exercise}
  \label{exercise:42k}
  \begin{enumerate}
    \item\label{enumerate:42k-a} Using~\autoref{exercise:42g} everything is already clear: the primes of~$A$ containing~$I$ form a subset of~$\Spec A$ and~$\phi^{-1}$ is an inclusion-preserving bijection, giving us the suggested picture.

    \item Using~\autoref{exercise:42h} everything is analogous.
  \end{enumerate}
\end{exercise}

\begin{exercise}
  The fiber of~$a\in\mathbb{C}$ corresponds to the preimage of the prime ideal (in this case: maximal ideal) defining~$a$, \ie, $(x-a)$. This obviously gives us~$y^2-a=(y-\sqrt{a})(y+\sqrt{a})$, hence the result.
\end{exercise}

\begin{exercise} % TODO is it?
  \begin{enumerate}
    \item This is a restatement of~\autoref{exercise:42k}\ref{enumerate:42k-a}.

    \item The Nullstellensatz gives us that all maximal ideals (\ie, points) of~$\mathbb{C}^n$ are exactly the ideals of the form~$(x_1-a_1,\ldots,x_m-a_m)$, which by~$\phi$ are mapped to the corresponding points of~$\mathbb{C}^n$.
  \end{enumerate}
\end{exercise}

\begin{exercise} % TODO fix this
  In the notation of~\autoref{exercise:42k}\ref{enumerate:42k-a} we have~$I=(x_1,\ldots,x_n)$ and~$B$ being~$\mathbb{Z}[x_1,\ldots,x_n]$. A point of~$\mathbb{A}_{\mathbb{F}_p}^n$ corresponds to a polynomial in~$n$~variables that is irreducible over~$\mathbb{F}_p$, which lies in the fiber over~$(p)$ because it is contained in the image of the induced ring map.
\end{exercise}

\begin{exercise}
  \label{exercise:42o}
  \begin{enumerate}
    \item We have that every prime ideal contains all nilpotents: if~$c$ is a nilpotent such that~$c^n=0$, we immediately find that~$c$ is an element of the prime ideal. The bijection is between primes of~$A/I$ and primes of~$A$ containing~$I$, but this latter set contains all primes, hence there is a bijection of the underlying sets.

    \item\label{enumerate:42o-b} Let's check the axioms. The sum of two nilpotents is again a nilpotent: take~$x$ and~$y$ nilpotents such that~$x^n=y^m=0$, we easily obtain
      \begin{equation}
        (x+y)^{n+m}=\sum_{i=0}^{n+m}\binom{n+m}{i}x^iy^{n+m-i}
      \end{equation}
      such that there always is a vanishing factor present in the expansion. Closed under multiplication is obviously true too: we have~$(bx)^n=b^nx^n=0$ for~$b\in B$.
  \end{enumerate}
\end{exercise}

\begin{exercise} % TODO copy
  I have seen this exercise in my Commutative algebra course. It might be a good idea to state it here as well.
\end{exercise}

\begin{exercise} % TODO do this
  I fail to find a decent argument.
\end{exercise}

\begin{exercise}
  A polynomial~$f\in k[x]$ corresponds to~$\sum_{k=0}^na_kx^k$. Now considering it over~$k[x,\epsilon]/(\epsilon^2)$ and ``evaluating'' it at~$x+\epsilon$ we find
  \begin{equation}
    f(x+\epsilon)=\sum_{k=0}^na_k(x+\epsilon)^k=\sum_{k=0}^na_k\left( x^n+nx^{n-1}\epsilon \right)
  \end{equation}
  because every term containing~$\epsilon^2$ is gone. If we move the first term of the inner sum to the left-hand side and dividing both sides by~$\epsilon$ (which isn't really possible, but for the sake of argument we can assume this), we see the fact~$(x^n)'=nx^{n-1}$.
\end{exercise}


\section{Visualing schemes I: generic points}

There are no exercises in this section.


\section{The underlying topological space of an affine scheme}

\begin{exercise}
  The~$x$\nobreakdash-axis corresponds to the ideal~$(y,z)$. This ideal contains the ideal~$(xy,yz)$, hence we have the inclusion of the axis in the vanishing set.
\end{exercise}

\begin{exercise}
  We have the obvious inclusion~$S\subseteq(S)$, so for the vanishing sets we have the opposite inclusion:
  \begin{equation}
    \vanishing{V}(S)=\left\{ [\mathfrak{p}]\in\Spec A\,|\, S\subseteq\mathfrak{p} \right\}\supseteq\left\{ [\mathfrak{p}]\in\Spec A\,|\, (S)\subseteq\mathfrak{p} \right\}=\vanishing\left( (S) \right).
  \end{equation}
  But if we take an element of the left-hand side, \ie, a prime ideal such that $S\subseteq\mathfrak{p}$ we can take an element of the generated ideal~$(S)$ and see that it is contained in~$\mathfrak{p}$ by the axioms of an ideal. I might have wasted too much words on this exercise.
\end{exercise}

\begin{exercise}
  \label{exercise:44c}
  \begin{enumerate}
    \item\label{enumerate:44c-a} We need to find vanishing sets such that their complements are~$\emptyset$ and~$\Spec A$. The vanishing set~$\vanishing(A)$ contains all of~$A$, hence its complement is empty. On the other hand,~$\vanishing(\left\{ 0 \right\})$ contains all prime ideals of~$A$, hence equals~$\Spec A$.

    \item This equality is obvious: for a point to be contained in the intersection, it must be contained in all ideals~$(I_i)_i$, but that means it is contained in the vanishing set of the sum of the ideals because this contains exactly all those elements.

    \item We obviously have~$\vanishing(I_1)\cup\vanishing(I_2)\subseteq\vanishing(I_1I_2)$ because~$I_1I_2\subseteq I_1,I_2$.

      For the other direction, consider~$\mathfrak{p}\in\Spec A\setminus(\vanishing(I_1)\cup\vanishing(I_2))$. That means there exist~$f\in I_1\setminus\mathfrak{p}$ and~$g\in I_2\setminus\mathfrak{p}$ such that~$fg\notin\mathfrak{p}$. But~$fg\in I_1I_2$, so~$\mathfrak{p}\nsupseteq I_1I_2$ so the other inclusion is proved.
  \end{enumerate}
\end{exercise}

\begin{exercise} % TODO better wording
  \label{exercise:44d}
  The fact that~$\sqrt{I}$ is an ideal is a restatement of~\autoref{exercise:42o}\ref{enumerate:42o-b}.
  
  Because~$I\subseteq\sqrt{I}$ we obviously have~$\vanishing(\sqrt{I})\subseteq\vanishing(I)$, so let's take~$\mathfrak{p}$ in the right-hand side. As~$\mathfrak{p}$ is prime and by definition of the radical of an ideal we only consider powers of elements, we have the opposite inclusion.

  The fact that~$\sqrt{\sqrt{I}}=\sqrt{I}$ follows from consecutive exponentiation. And if~$f^m\in\mathfrak{p}$ for some~$m$ we have~$f\in\mathfrak{p}$ by primeness.
\end{exercise}

\begin{exercise}
  Consider the scenario of two ideals, hence we wish to prove
  \begin{equation}
    \sqrt{I\cap J}=\sqrt{I}\cap\sqrt{J}.
  \end{equation}
  Because taking the radical preserves inclusion we have the inclusion from left to right as~$I\cap J\subseteq I,J$ and therefore~$\sqrt{I\cap J}\subseteq\sqrt{I},\sqrt{J}$ so~$\sqrt{I\cap J}\subseteq\sqrt{I}\cap\sqrt{J}$.
  
  Now take~$x\in\sqrt{I}\cap\sqrt{J}$, there must exist an~$n$ such that~$x^n\in\sqrt{I}$ and~$m$ such that~$x^m\in\sqrt{J}$. But now~$x^{n+m}\in I\cap J$, so~$x\in\sqrt{I\cap J}$.

  The general result follows by induction.
\end{exercise}

\begin{exercise}
  \label{exercise:44f}
  Perform it on~$A/I$. The ideal~$I$ is sent to~$0$ under the quotient map, hence the nilradical~$\nil(A/I)$ is the intersection of all prime ideals containing~$I$, but this corresponds to~$\sqrt{I}$ in~$A$.
\end{exercise}

\begin{exercise}
  By~\autoref{exercise:42j} closed points pull back to closed points. Now a closed set is the finite union of closed points, which under an inverse map is preserved.
\end{exercise}

\begin{exercise} % TODO finish this
  \label{exercise:44h}
  \begin{enumerate}
    \item By~\autoref{exercise:42g} we have that~$\Spec B/I$ corresponds to the vanishing set~$\vanishing(I)$, hence it is closed by definition of the Zariski topology.

      Analogously, the vanishing set~$\vanishing((f))$ described the complement of~$\Spec S^{-1}B$ in this case.

      This construction doesn't work for every localization as is clear from Figure~4.5 and~\autoref{exercise:42d}. The complement of the ``shred of~$\mathbb{A}_{\mathbb{C}}^2$'' doesn't fulfill the conditions for a closed set described there and in Example~7 of that section.

    \item By an argument on the vanishing sets we easily obtain the statement. Maybe I should put this down in symbols.
  \end{enumerate}
\end{exercise}

\begin{exercise}
  \label{exercise:44i}
  By definition we have
  \begin{equation}
    \vanishing(I)=\left\{ [\mathfrak{p}]\in\Spec A\,|\,I\subseteq[\mathfrak{p}] \right\}
  \end{equation}
  and~$f$ vanishes on this set if and only if it is contained in the prime ideals that constitute the points of~$\vanishing(I)$. But using~\autoref{exercise:44f} this is exactly equivalent to~$f\in\sqrt{I}$.
\end{exercise}

\begin{exercise}
  By~\autoref{exercise:44h} we have the subspace topology on~$\Spec k[x]_{(x)}$ so we need to state what the subspace is. By using~\autoref{exercise:42h} we obtain the affine line with the origin (corresponding to~$(x)$) removed.
\end{exercise}


\section{A base of the Zariski topology on \texorpdfstring{$\Spec A$}{Spec A}: distinguished open sets}

\begin{exercise}
  \label{exercise:45a}
  Using the hint we find
  \begin{equation}
    X\setminus\vanishing(S)=\left\{ [\mathfrak{p}]\in\Spec A\,|\,S\nsubseteq[\mathfrak{p}] \right\}
  \end{equation}
  and~$I\nsubseteq\mathfrak{p}$ if there is an~$f\in S$ such that~$f\notin\mathfrak{p}$, so~$\distinguished(f)\subseteq X\setminus\vanishing(S)$ in this case. We obtain~$X\setminus\vanishing(S)=\bigcup_{f\in S}\distinguished(f)$ as desired.
\end{exercise}

\begin{exercise} % TODO do this
  \label{exercise:45b}
  I have to write down a nice answer.
\end{exercise}

\begin{exercise}
  \label{exercise:45c}
  If~$\bigcup_{j\in J}\distinguished(f_j)=\Spec A$ we have~$(f_j)_{j\in J}=A$ by~\autoref{exercise:45b}, which means there is a finite subset~$J'$ of the index set~$J$ such that~$\sum_{j\in J'}a_jf_j=1$, so~$(f_j)_{j\in J'}=A$, such that~$\bigcup_{j\in J'}\distinguished(f_j)=\Spec A$.
\end{exercise}

\begin{exercise}
  We have
  \begin{equation}
    \begin{aligned}
      \distinguished(f)\cap\distinguished(g)&=\left\{ [\mathfrak{p}]\in\Spec A\,|\,f\notin\mathfrak{p}\wedge g\notin\mathfrak{p} \right\} \\
      &=\left\{ [\mathfrak{p}]\in\Spec A\,|\,fg\notin\mathfrak{p} \right\} \\
      &=\distinguished(fg)
    \end{aligned}
  \end{equation}
  because~$\mathfrak{p}$ is prime, so~$fg$ cannot be an element of it unless at least one of the factors is.
\end{exercise}

\begin{exercise} % TODO second equivalence
  \label{exercise:45e}
  By taking complements we can reduce this to~\autoref{exercise:44i}.
\end{exercise}

\begin{exercise}
  If~$f\in\mathfrak{N}$ we have~$f^n=0$ for some~$n$, which by~\autoref{exercise:44c}\ref{enumerate:44c-a} (this is more background information than the actual reason) and~\autoref{exercise:45e} gives~$\distinguished(f)\subseteq\distinguished(0)=\emptyset$.
\end{exercise}


\section{Topological definitions}

\begin{exercise}
  If there were a nonempty non-dense open subset, take its complement as~$Y$ and its closure as~$Z$ in the definition of irreducibility. We have a contradiction.
\end{exercise}

\begin{exercise}
  Take~$X=Y\cup Z$ as in the definition. The generic point~$(0)$ (because~$A$ is integral) should be contained in at least one of these. But by definition of a vanishing set we have that \emph{all} points of~$\Spec A$ will be contained in this vanishing set, hence~$Y$ or~$Z$ equals~$X$.
\end{exercise}

\begin{exercise}
  If~$\left\{ x \right\}$ is a closed subset, it corresponds to a vanishing set. But this vanishing set equals~$\left\{ x \right\}$ if and only if the only point it contains is the prime ideal (corresponding to)~$x$, which happens if and only if it is a maximal ideal. If it wasn't maximal, we'd have by application of Zorn's lemma a maximal ideal containing it and maximal ideals are prime, contradiction the fact that the set is a singleton.
\end{exercise}

\begin{exercise}
  \begin{enumerate}
    \item By~\autoref{exercise:45a} the distinguished opens form a basis for the Zariski topology and by~\autoref{exercise:45c} we can reduce a covering using distinguished opens to a finite covering. Replace the general covering~$\bigcup_{i\in I}U_i$ by the covering where every~$U_i$ is replaced by the (arbitrary) union of distinguished opens. Take the finite subcover of these distinguished opens and take the original open sets that corresponded to the distinguished opens, these are bigger, hence constitute a finite subcover themselves.

    \item The ideal~$\mathfrak{m}=(x_1,x_2,\ldots)$ is a maximal ideal as~$A/\mathfrak{m}=k$, a field. The complement of~$\vanishing(\mathfrak{m})$ can be covered using distinguished sets~$\distinguished((x_i))_{i\in\mathbb{N}}$, this covering doesn't admit a finite subcovering.
  \end{enumerate}
\end{exercise}

\begin{exercise}
  \begin{enumerate}
    \item Given a cover of~$X$, consider the induced coverings of the subspaces, take finite subcovers there and take the finite union of these covers.

    \item Given a cover of a closed subset of a quasicompact space, we can add the complement to every open set used in the covering, which gives us by definition of the subspace topology (all open sets in the subspace arise by intersecting with the subspace) a covering of the big space using open sets. Apply the quasicompactness condition here to obtain a finite covering and remove the complement of the closed subset, resulting in a finite subcover .
  \end{enumerate}
\end{exercise}

\begin{exercise}
  \label{exercise:46f}
  By definition~$\vanishing(\mathfrak{p})=\left\{ [\mathfrak{q}]\in\Spec A\,|\, \mathfrak{p}\subseteq\mathfrak{q}\right\}$, while the closure of~$\left\{ [\mathfrak{p}] \right\}$ contains all points in~$\Spec A$ that are contained in the point~$[\mathfrak{p}]$, we see that these definitions are equivalent.
\end{exercise}

\begin{exercise}
  This is obvious from~\autoref{exercise:46f} and the definition:~the point~$x$ is represented by~$[\mathfrak{p}]=[(y-x^2)]$ which is a prime ideal and we obtain
  \begin{equation}
    \overline{\left\{ x \right\}}=\overline{\left\{ [\mathfrak{p}] \right\}}=\vanishing(\mathfrak{p})=\vanishing(y-x^2)=K.
  \end{equation}
\end{exercise}

\begin{exercise} % TODO do this
  \begin{enumerate}
    \item This is too tricky for the moment, I'll come back to this later.

    \item This is too tricky for the moment, I'll come back to this later.
  \end{enumerate}
\end{exercise}

\begin{exercise}
  Assume for the sake of contradiction that we have an infinite descending chain. Using~4.4.3 the first closed subset that is not the entire space\footnote{We've assumed an infinite descending chain. Just repeating the whole space isn't quite infinite.} is built using a finite number of curves and closed points. The next set (ignoring equalities, of which there are only finitely many between each real step) contains one curve or one point less (that cannot lie on one of the curves!). This process can only be repeated a finite number of times, ending in~$\emptyset$, a contradiction on the infiniteness assumption.
\end{exercise}

\begin{exercise}
  Assume~$A$ is Noetherian and take an ideal~$I$ that is not finitely generated, \ie, we have~$I=(x_1,x_2,\ldots)$. Now construct the chain
  \begin{equation}
    (x_1)\subsetneqq(x_1,x_2)\subsetneqq(x_1,x_2,x_3)\subsetneqq\ldots
  \end{equation}
  and this would constitute an infinitely ascending chain, contradiction. At some point the ideal~$I$ should pop up and we have a finite set of generators.

  Assume every ideal of~$A$ is finitely generated, but~$A$ is not Noetherian. Take an infinite ascending chain of ideals, the union is again an ideal. This ideal is finitely generated, so take the index for which each of the generators is contained in the ideal at that position of the chain. We have equality for all subsequent ideals, contradicting the lack of Noetherianness.
\end{exercise}
