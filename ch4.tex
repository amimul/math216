\section{Toward schemes}

\begin{exercise} % TODO do this
  I should be ashamed of myself, not being able to answer this question.
\end{exercise}

\begin{exercise} % TODO do this
  I should be ashamed of myself, not being able to answer this question.
\end{exercise}


\section{The underlying set of affine schemes}

\begin{exercise}
  \label{exercise:42a}
  \begin{enumerate}
    \item\label{enumerate:42a-a} So we're looking for the prime ideals of~$\Spec k[\epsilon]/(\epsilon^2)$. But these correspond to the prime ideals of~$\Spec k[\epsilon]$ containing~$(\epsilon^2)$. Now the only prime ideal of this form is~$(\epsilon)$. This corresponds to the polynomials in~$\epsilon$ with no constant term. If there would be a constant term, \ie, something of the form~$a+b\epsilon$ it would be invertible modulo~$\epsilon^2$ using a geometric series. There is only one point.

      Notice that~$\Spec k[\epsilon]/(\epsilon)$ is not an integral domain:~$\epsilon^2$ is contained in~$(0)$ yet~$\epsilon\notin(0)$.

    \item\label{enumerate:42a-b} By commutative algebra the prime ideals of the localization correspond to the prime ideals of~$\Spec k[x]$ not containing~$(x)$. So the set~$\Spec k[x]_{(x)}$ corresponds to~$\Spec k[x]\setminus\left\{ (x) \right\}$ because there is only one prime ideal containing~$x$ namely~$(x)$: if there would be another one we could reduce it to a constant ending up the whole ring, a contradiction.
  \end{enumerate}
\end{exercise}

\begin{exercise}
  Using the discriminant we obtain the two roots of the quadratic which look like
  \begin{equation}
    x_{1,2}=\frac{-a\pm\sqrt{a^2-4b}}{2}
  \end{equation}
  where~$a^2-4b<0$. Now using operations of~$\mathbb{R}$ we can reduce this to~$i$.
\end{exercise}

\begin{exercise}
  This set corresponds to all polynomials that are irreducible over~$\mathbb{Q}$. There are the obvious points~$(x-a)$ where~$a\in\mathbb{Q}$, but all roots of polynomials are present too but they are glued together by the corresponding Galois actions. It corresponds to the identification of roots in the algebraic closure~$\mathbb{Q}^{\alg}$.
\end{exercise}

\begin{exercise}
  Suppose~$\mathfrak{p}$ is a prime ideal that is not a principal ideal. Take two essential generators~$f(x,y)$ and~$g(x,y)$ (\ie, with not all factors of one contained in the other). This must be possible because otherwise we wouldn't have a principal ideal: one can be written as a product of the other with a polynomial containing the missing factors. Now because~$\mathfrak{p}$ is prime we can remove all common factors.

  By applying the Euclidean algorithm in~$\mathcal{C}(x)[y]$ we can find a polynomial in the variable~$x$ contained in~$(f(x,y),g(x,y))\subseteq\mathfrak{p}$, which by the algebraic closedness of~$\mathcal{C}$ reduces to a linear factor~$(x-a)$ contained in~$\mathfrak{p}$ and analogously~$(y-b)\in\mathfrak{p}$.

  Obviously any principal ideal must be generated by an irreducible polynomial. So having reduced all non-principal ideals to ideals of the form~$(x-a,y-b)$ we have finished the proof.
\end{exercise}

\begin{exercise}
  The first maximal ideal is~$(x^2+y^2-4,x-y)$ while the second is~$(x^2+y^2-4,x+y)$. The residue fields are~$\mathbb{Q}(\sqrt{2})$ in both cases: substituting the second generator in the first yields this result.
\end{exercise}

\begin{exercise} % TODO do this
  I'm still wondering whether my answer is correct.
\end{exercise}

\begin{exercise}
  I have used this fact in~\autoref{exercise:42a}\ref{enumerate:42a-a}. It boils down to
  \begin{equation}
    A/J\cong (A/I)/(J/I) 
  \end{equation}
  where~$I\subseteq J$ are prime ideals of~$A$, and this is equivalent to~$\overline{J}$ being prime in~$A/I$.
\end{exercise}

\begin{exercise} % TODO rewrite this
  I have used this fact in~\autoref{exercise:42a}\ref{enumerate:42a-b}. A prime ideal of~$A$ that contains an element of~$S$ will the whole become~$S^{-1}A$ under localization. A prime ideal of~$A$ disjoint of~$S$ remains prime because the product~$(p_1/s_1)(p_2/s_2)$ is in the prime ideal if and only if~$(p_1p_2)/(s_1s_2)$ is in the prime ideal, but we can multiply with~$s_1s_2$ as this is by multiplicativity of~$S$ not a member of the prime ideal. Now we have reduced it to~$p_1p_2$ in the prime ideal.
\end{exercise}
