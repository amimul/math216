\section{The structure sheaf of an affine scheme}

\begin{exercise}
  We have~$\vanishing(f)=\left\{ [\mathfrak{p}]\in\Spec A\,|\,f\in\mathfrak{p} \right\}$, functions not vanishing outside~$\vanishing(f)$ are by~\autoref{exercise:45e} correspond to~$g$ such that~$f^n\in(g)$. Therefore the localisation~$\mathcal{O}_{\Spec A}(\distinguished(f))$ consists of fractions such that the denominator is an element of the principal ideal~$(g)$ such that~$f^n\in(g)$. Remark that this is obviously a multiplicative set and the isomorphism is given by the result of~\autoref{exercise:45e}.
\end{exercise}

\begin{exercise}
  This really boils down to judiciously replacing~$A$ by~$A_f$. We obtain:
  \begin{quote}
    We check identity on the base. Suppose that~$\Spec A_f=\bigcup_{i\in I}\distinguished(f_i)$ where~$i$ runs over some index set~$I$. Then there is some finite subset of~$I$, which we name~$\left\{ 1,\ldots,n \right\}$, such that~$\Spec A_f=\bigcup_{i=1}^n\distinguished(f_i)$, \ie,~$(f_1,\ldots,f_n)=A_f$ (quasicompactness of~$\Spec A_f$, \autoref{exercise:45c}).

    Suppose we are given~$s\in A_f$ such that~$\res_{\Spec A_f,\distinguished(f_i)}s=0$ in~$A_{f_i}$ for all~$i$. We wish to show that~$s=0$. The fact that~$\res_{\Spec A_f,\distinguished(f_i)}s=0$ in~$A_{f_i}$ implies that there is some~$m$ such that for each~$i\in\left\{ 1,\ldots,n \right\}$, ~$f_i^ms=0$. Now~$(f_1^m,\ldots,f_n^m)=A_f$, for example, from
    \begin{equation}
      \Spec A_f=\bigcup_{i=1}^n\distinguished(f_i)=\bigcup_{i=1}^n\distinguished(f_i^m),
    \end{equation}
    so there are~$r_i\in A_f$ with~$\sum_{i=1}^nr_if_i^m=1$ in~$A_f$, from which
    \begin{equation}
      s=\left( \sum_{i=1}^nr_if_i^m \right)s=\sum_{i=1}^nr_i(f_i^ms)=0.
    \end{equation}
    Thus we have checked the ``base identity'' axiom for~$\Spec A_f$.
  \end{quote}
\end{exercise}

\begin{exercise}
  \label{exercise:51c}
  Again, replacing~$A$ with~$A_f$, open covers of~$\Spec A$ with open covers of the open subspace~$\Spec A_f$ and copying the entire proof suffices. Only three occurences of~$A$ should be replaced with~$A_f$
\end{exercise}

\begin{exercise}
  \label{exercise:51d}
  We have to redo Theorem~5.1.2, but now for this more general construction\todo{\autoref{exercise:51d}: prove that~$\tilde{M}$ is a sheaf}.
\end{exercise}


\section{Visualing schemes II: nilpotents}

There are no exercises in this section.


\section{Definition of schemes}

\begin{exercise}
  If~$f\colon A'\to A$ is a isomorphism of rings, the induced affine schemes are obviously isomorphic too: the underlying spaces are homeomorphic (the vanishing sets are ``equal'' by the isomorphism) and we can put the induced isomorphism for every localization that occurs in the corresponding sheaves.

  If~$f\colon\Spec A\to\Spec A'$ is an isomorphism of affine schemes, the rings of global sections are isomorphic. Now this is a bijection because the rings~$A$ and~$A'$ determine everything there is to know about these ringed spaces and their isomorphisms.
\end{exercise}

\begin{exercise}
  We have the homeomorphism between~$\distinguished(f)$ and~$\Spec A_f$ by Section~4.5. Now the sheaves are isomorphic too: the restriction~$\mathcal{O}_{\Spec A}|_{\distinguished(f)}$ has~$\sections(\distinguished(f),\mathcal{O}_{\Spec A})=A_f$ as global ring of sections which induces the desired isomorphism.
\end{exercise}

\begin{exercise}
  \label{exercise:53c}
  Consider~$p\in U\subseteq X$. Take~$V$ a neighbourhood of~$p$ such that~$(V,\mathcal{O}_X|_V)$ is an affine scheme, \ie, is isomorphic to~$(\Spec A,\mathcal{O}_{\Spec A})$ for some ring~$A$. Now consider the restriction of~$\mathcal{O}_{\Spec A}$ to~$U\cap V$ which is again an affine scheme\todo{\autoref{exercise:53c}: an open subscheme of an affine scheme is again affine, but formalize this please}.
\end{exercise}

\begin{exercise}
  By definition of a scheme we have for~$U\subseteq X$ open that~$(U,\mathcal{O}_X|_U)$ is an affine scheme, which induces the Zariski topology on~$U$. Now all the restrictions are given from that point, so we just have to take arbitrary unions of open sets regarding every open~$U$ to obtain all open sets in~$X$.
\end{exercise}

\begin{exercise}
  \begin{enumerate}
    \item Everything follows from~\autoref{exercise:46t}: we have the homeomorphism~$\coprod_{i=1}^n\Spec A_i\to\Spec\prod_{i=1}^nA_i$ which gives us that the finite disjoint union is isomorphic to the spectrum of a finite product of rings, which is an affine scheme, just take~$A=\prod_{i=1}^nA_i$ in the definition of the affine scheme~$\coprod_{i=1}^n\Spec A_i$.

    \item By~\autoref{exercise:46d}\ref{enumerate:46d-a} we have that affine schemes are quasicompact, but the infinite disjoint union can be covered by taking the (affine) opens~$\Spec A_i$, which are by definition disjoint but open. We've obtained an infinite cover that cannot be reduced.\qedhere
  \end{enumerate}
\end{exercise}

\begin{exercise}
  The stalk in~$[\mathfrak{p}]$ is defined as the direct limit of the sections over all open subsets containing~$[\mathfrak{p}]$. These open sets are generated by the~$\distinguished(f)$ for~$f\notin\mathfrak{p}$. The sections over these sets are the localizations~$A_f$. In the direct system these will be identified accordingly, but because~$\mathfrak{p}$ is prime, one will never obtain a localization at an element of~$\mathfrak{p}$. Yet all elements in~$A\setminus\mathfrak{p}$ \emph{will} be inverted.
  
  The localization at a prime ideal on the other hand is the localization at the complement~$A\setminus\mathfrak{p}$. We obtain the desired result.
\end{exercise}
