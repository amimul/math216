\section{Motivating example: the sheaf of differentiable functions}

\begin{exercise}
  As every element of~$\mathcal{O}_p\setminus\mathfrak{m}$ is nonzero in a neighbourhood of~$p$ we can restrict an element such that it is invertible there, a property which is preserved when taking the stalk. Hence the germ of a non-vanishing function is invertible and~$\mathfrak{m}$ is invertible.
\end{exercise}

\begin{exercise} % TODO find out what is asked for
  I don't really have a differential geometry background and I fail to see what should be proved. But I should revisit this exercise later.
\end{exercise}


\section{Definition of sheaf and presheaf}

\begin{exercise}
  A functor assigns an object in~$\category{Sets}$ to every object in the category of open sets of the topological space~$X$. As an inclusion of sets is reflected as a morphism of the two open sets involved, this is translated to a morphism of sets in the codomain category\footnote{I have never seen this terminology, don't shoot me if I missed some more obvious wording.}. Identities are preserved by functors, so~$\res_{U,U}=\identity_{\mathcal{F}(U)}$ by definition. For the commutativity of restriction maps, this is by the \emph{contravariance} of the functor.
\end{exercise}

\begin{exercise}
  \begin{enumerate}
    \item The presheaf axioms are (trivially) true by restriction of functions. Yet it is not a sheaf, take~$x\mapsto|x|$. On~$\ball(0,n)$ the function is bounded by~$n$, we can take~$\mathbb{C}=\bigcup_{n\in\mathbb{N}}\ball(0,n)$ and glue together a function on all of~$\mathbb{C}$ yet it is not bounded hence not a section over~$\mathcal{C}$ of this sheaf.

    \item Inspired by \href{http://math.stackexchange.com/questions/38423/presheaf-which-is-not-a-sheaf-holomorphic-functions-which-admit-a-holomorphic}{\texttt{math.stackexcange.com}} (and my complex analysis course) the idea is to circle a zero, ending up with a multiple-valued function, which is clearly not a section of the sheaf.

      Take~$f\colon z\mapsto z$ on the annulus~$U=\left\{ 1-\epsilon<|z|<1+\epsilon \right\}$. By the Cauchy integral formula we have
      \begin{equation}
        \oint_{\mathclap{|z|=1}}\frac{g'(z)}{g(z)}\dd z\in2\pi\mathrm{i}\mathbb{Z}
      \end{equation}
      where~$g$ is holomorphic, without zeroes on the unit circle. Hence for~$f$ we obtain~$2\pi\mathrm{i}$.

      But if~$f=g^2$ we would obtain
      \begin{equation}
        \oint_{\mathclap{|z|=1}}\frac{f'(z)}{f(z)}\dd z=2\oint_{\mathclap{|z|=1}}\frac{g'(z)}{g(z)}\dd z\in4\pi\mathrm{i}\mathbb{Z}
      \end{equation}
      which is a contradiction on the value we previously obtained. Now~$f$ can be patched together from holomorphic functions that admit a square root, as long as the open set doesn't wind around zero: take open balls to cover~$U$ and we're done.
  \end{enumerate}
\end{exercise}

\begin{exercise}
  As we have maps~$\mathcal{F}(\bigcup U_i)\to\mathcal{F}(U_i)$ (or arbitrary unions of~$U_i$) this is a limit: the arrows from our desired object map \emph{to} all the objects as described in~2.4.4.
\end{exercise}

\begin{exercise}
  \label{exercise:32d}
  \begin{enumerate}
    \item Such functions are defined in their points, if all the restrictions agree for a covering, the functions are obviously equal. The identity axiom is satisfied. But given a covering on which all restrictions agree we can just paste together a global function, hence the gluability axiom is satisfied~too.

      Glueing functions preserves their extra properties: these are all defined in a neighbourhood of a point, hence valid in an open set and therefore lift to the global function.

    \item Analogous.
  \end{enumerate}
\end{exercise}

\begin{exercise}
  The presheaf axioms are trivially satisfied, local functions restrict easily. The identity axiom is obvious too. For the gluability: observe that the compatibility boils down to defining a function on the \emph{connected components} of the covering because sections are constant if there is a non-empty intersection and therefore this axiom is satisfied too.
\end{exercise}

\begin{exercise}
  \label{exercise:32f}
 The key idea is ``a function is continuous'' if and only if ``locally a function is continuous''. Hence restriction of continuous functions is well defined and the commutativity of the restriction triangle is immediate. Now for the identity axiom: equality of functions is checked in the points and as all restrictions agree the global sections must agree. Likewise we can check the gluability axiom.
\end{exercise}

\begin{exercise}
  \begin{enumerate}
    \item The sections of~$f$ over~$U$ form a subset of the sections of the sheaf from~\autoref{exercise:32f} over~$U$. The restrictions are trivially fine, the identity axiom follows from the previous exercise too and gluability is obviously true as well as~$f\circ s=\identity|_U$ holds for the pointwise glueing of a section.

    \item There is nothing to add to the arguments of~\autoref{exercise:32d}. The set of sections carries a natural pointwise abelian group structure.
  \end{enumerate}
\end{exercise}

\begin{exercise}
  The presheaf axioms are satisfied because~$f$ is continuous hence the inverse~$f^{-1}$ maps open sets to open sets. The presheaf axioms are satisfied on the open sets in the domain of the map and the presheaf axioms for~$f_*\mathcal{F}$ are a subset (sketchy wording) of those for~$\mathcal{F}$.

  The sheaf axioms are satisfied because inverse maps are compatible with unions and intersections. As for the presheaf axioms: everything is transferred.
\end{exercise}

\begin{exercise}
  Using the definition of the direct limit and the construction as described in~\autoref{exercise:23c} we see that \emph{less} relations have to be quotiented, as the direct system defining~$(f_*\mathcal{F})_y$ is contained in the direct system defining~$\mathcal{F}_p$.
\end{exercise}

\begin{exercise}
  As germs in the stalk~$\mathcal{F}_x$ are equivalence classes of functions defined in the neighbourhood modulo equality on a (smaller) neighbourhood, we can define the~$\mathcal{O}_{X,x}$\nobreakdash-module structure on~$\mathcal{F}_x$ by defining it on representatives of these classes. By commutativity of~(3.2.12.1) this is well defined. The actual checking of the structure is straightforward and familiar.
\end{exercise}


\section{Morphisms of presheaves and sheaves}

\begin{exercise}
  The induced morphism is given by applying the morphism to representatives of equivalence classes of the stalks. By commutativity of the diagram in the definition of (pre)sheaf morphisms this is correct.
\end{exercise}

\begin{exercise}
  To every sheaf~$\mathcal{F}$ in~$\category{Sets}_X$ there is a (unique) associated sheaf~$f_*(\mathcal{F})$ in~$\category{Sets}_Y$. Obviously~$f_*(\identity_\mathcal{F})=\identity_{f_*(\mathcal{F})}$ and we have~$f_*(h\circ g)$ with~$g\colon\mathcal{F}\to\mathcal{G}$ and~$h\colon\mathcal{G}\to\mathcal{H}$ by applying this to every open set~$U$.
\end{exercise}

\begin{exercise}
  The presheaf axioms are easy: the associated restriction is just the obvious restriction of a function~$f\colon\mathcal{F}(U)\to\mathcal{G}(U)$ to~$f|_V\colon\mathcal{F}(V)\to\mathcal{G}(V)$ in~$\SheafHom(\mathcal{F},\mathcal{G})(V)$ for~$V\subseteq U$.

  The identity axiom for~$\SheafHom$ follows from the separatedness of~$\mathcal{G}$. Take a cover~$(U_i)_i$ of~$U$, any open subset~$V$ of~$U$ and a section~$s\in\mathcal{F}(V)$. Now we have~$f(V)(s)|_{U_i\cap V}=g(V)(s)|_{U_i\cap V}$ (which are elements of the sheaf~$\mathcal{G}$) for the covering~$(U_i\cap V)_i$ of~$V$. As~$\mathcal{G}$ is assumed to be separated we obtain~$f(V)(s)=g(V)(s)$. As this holds for any~$V$ and all sections~$s$, we have that~$\SheafHom$ is a separated presheaf.

  Now consider a family~$(f_i)_i$ of compatible sections (which are morphisms of sets of sections) on a cover~$(U_i)_i$ of~$U$. Like the case of the identity axiom, we consider~$V\subseteq U$ open and~$s\in\mathcal{F}(V)$. This gives
  \begin{equation}
    f_i(U_i\cap V)(s|_{U_i\cap V})\in\mathcal{G}(U_i\cap V).
  \end{equation}
  Now glue together~$f(V)(s)$ in~$\mathcal{G}(V)$. This defines a morphism of presheaves, by separatedness of~$\mathcal{G}$ (the compatibility of restrictions is induced in a unique way).

  If~$\mathcal{G}$ is a sheaf of abelian groups, morphisms~$\mathcal{F}\to\mathcal{G}$ carry a natural group structure, by pointwise addition. The neutral element is the zero map.
\end{exercise}

\begin{exercise} % TODO expand
  \begin{enumerate}
    \item By definition
      \begin{equation}
        \SheafHom\left( \underline{\left\{ p \right\}},\mathcal{F} \right)=\Mor\left( \underline{\left\{ p \right\}}|_U=\left\{ p \right\},\mathcal{F}|_U \right).
      \end{equation}
      As there is a unique map~$f_x\colon\left\{ p \right\}\to\mathcal{F}|_U:p\mapsto x$ for every element of~$x\in\mathcal{F}|_U$ (hence a bijection of sets) we have obtained an isomorphism of sheaves.

    \item\label{enumerate:exercise-33d-b} Analogously we now have a map~$f_g\colon\underline{\mathbb{Z}}(U)=\mathbb{Z}\to\mathcal{F}(U)$ such that~$1\mapsto g$, defined for every~$g\in\mathcal{F}(U)$. Now take sections~$g$ and~$g'\in\mathcal{F}(U)$. We obtain~$f_{g+g'}=f_g+f_{g'}$ because the image of~$1$ defines everything (\ie,~$\mathbb{Z}$ is the free group generated by~$\left\{ 1 \right\}$).

    \item Analogous to~\ref{enumerate:exercise-33d-b}.
  \end{enumerate}
\end{exercise}

\begin{exercise}
  In the situation of the diagram as given in the notes, we have injective maps~$\ker_{\textrm{pre}}\phi(V)\to\mathcal{F}(V)$ for every~$V$ open. Define~$\res^{\ker}_{V,U}$ by mapping~$g\in\ker_{\textrm{pre}}\phi(V)$ to the preimage of~$\res_{V,U}$ under the injection maps. As both maps are injective, this preimage is a well-defined element. We have found our induced restriction map. It is unique by commutativity of the square and the injectivity: any two maps such that~$f\circ g_1=f\circ g_2$ imply~$g_1=g_2$.

  We now check the conditions required for a presheaf. Obviously~$\res^{\ker}_{U,U}$ is the identity map and the commutativity of the restriction triangle follows from the uniqueness of the restriction maps.
\end{exercise}

\begin{exercise} % TODO draw diagram and finish exercise
  I promise to draw this diagram first thing tomorrow, right now I'm not in the mood.
\end{exercise}

\begin{exercise}
  \label{exercise:33g}
  A morphism between sheaves is defined as a morphism of the category over which we are considering the sheaf for \emph{every} open set. We now restrict ourselves to a specific open set, hence the morphisms (the only nontrivial part of a functor's definition) are preserved. 

  This functor is exact because the exactness of a sequence of sheaves if checked over open sets by the fact that all abelian-categorical notions are verified ``open set by open set''. And we only consider a specific open set, over which the exactness is given.
\end{exercise}

\begin{exercise}
  The abelian-categorial notions are verified ``open set by open set'', hence we consider a family of functors as in~\autoref{exercise:33g} indexed by all open sets in the topology on~$X$. The equivalence follows.
\end{exercise}

\begin{exercise} % TODO finish
  The uniqueness of the induced restriction map and the injectivity for every open set give us the identity axiom: if all restrictions are equal we can just move everything one morphism to the right and identify things there. The same holds for the gluability axiom: given local~$f_i$, we move everything using the injectivity to the sheaf's object~$\mathcal{F}(U_i)$ and lift things to~$\mathcal{F}(U)$. Now we don't now yet that this section lives in~$\ker_{\textrm{pre}}\phi$, but if it wouldn't, there needs to be an open set for which the image under~$\phi$ is nonzero. % TODO
\end{exercise}

\begin{exercise} % TODO finish
  The inclusion~$\underline{\mathbb{Z}}\hookrightarrow\mathcal{O}_X$ is obvious: constant functions are holomorphic. Under the mapping~$f\mapsto\exp(2\pi\mathrm{i}f$ the kernel are exactly those functions such that~$\exp(2\pi\mathrm{i}f)$ is~$1$ (the neutral element in this multiplicative group). This is true for all constant integer functions:~$2\pi\mathrm{i}n$ evaluates under exponentiation to~$1$. As~$\mathcal{F}$ is the presheaf of functions admitting a holomorphic logarithm, they must come from the exponentiation of a holomorphic function contained in~$\mathcal{O}_X$ and we have the surjectivity of~$\mathcal{O}_X\twoheadrightarrow\mathcal{F}$.

  The failure of the sheaf axioms is for tomorrow.
\end{exercise}


\section{Properties determined at the level of stalks, and sheafification}

\begin{exercise}
  Assume the map is not injective, \ie, there are sections~$f$ and~$g$ in~$\mathcal{F}(U)$ such that the product of their values in the stalks in~$U$ is equal. Every value arises by construction from a section on a neighbourhood, hence for the intersection of these neighbourhoods the restrictions of~$f$ and~$g$ have to agree. Now use the identity axiom because~$\mathcal{F}$ is a separated presheaf, which leads to~$f=g$, a contradiction.
\end{exercise}
