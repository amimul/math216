\section{Motivation}

There are no exercises in this section.


\section{Categories and functors}

\begin{exercise}
  \begin{enumerate}
    \item The elements of the group(oid) correspond to the morphisms. As every morphism is an isomorphism, we can only compose morphisms on the same object. Now in case of a single object, all axioms for a group are satisfied, as isomorphisms lead to inverse elements.

    \item Take the category of a group and copy the unique object, together with all its morphisms. Voil\`a, a groupoid.

      A natural example of groupoids is the fundamental groupoid of a topological space. You cannot combine loops that have different base points.
  \end{enumerate}
\end{exercise}

\begin{exercise}
  By definition of \emph{invertible element of~$\Mor(A,A)$} the automorphisms form a group: we get composition and associativity from the category and the identity and inverse from our choice of elements.
  
  In case of~$\category{Sets}$ the automorphisms are the permutations of the set, \ie, bijections. In case of~$\category{Vec}_k$ the automorphisms are bijective linear self-maps.

  By conjugation, isomorphic objects have isomorphic automorphism groups.
\end{exercise}

\begin{exercise} % TODO I should probably do this
  Linear algebra exercise. I will do this one if I feel inspired.
\end{exercise}

\begin{exercise}
  A basis for a finite-dimensional vectorspace has a well-defined cardinality, defining its dimension. So the inverse functor~$\category{f.d.\ Vec}_k\to\mathcal{V}$ maps an~$n$\nobreakdash-dimensional vectorspace~$V$ to~$k^n$, while every linear map between finite-dimensional vectorspaces can be written against a choice of bases. As we were allowed to pick a basis for each vectorspace simultaneously, every linear map admits by linear algebra magic with tons of indices a representation as a matrix.
\end{exercise}


\section{Universal properties determine an object up to unique isomorphism}

\begin{exercise}
  \label{exercise:23a}
  Take~$A$ and~$B$ initial objects. As there exist (unique) morphisms~$A\to B$ and~$B\to A$ by definition of an initial object, we can compose them, obtaining morphisms~$A\to A$ and~$B\to B$. But the identity is another candidate for this morphisms, so by uniqueness of the morphisms~$A$ and~$B$ are isomorphic.

  The proof for final objects is completely the same.
\end{exercise}

\begin{exercise}
  \begin{description}
    \item[$\category{Sets}$] The initial object is the empty set~$\emptyset$, the singleton~$\left\{ x \right\}$ is the final object (all singletons are isomorphic as stated before in~\autoref{exercise:23a}).
    \item[$\category{Rings}$] As the image of~$1\in\mathbb{Z}$ determines the entire ring morphism, the ring of integers~$\mathbb{Z}$ is the initial object. The final object is the trivial ring (in which~$0=1$).
    \item[$\category{Top}$] The initial object is the empty set~$\emptyset$ equipped with the topology consisting of~$1$~open set, the final object is the singleton equipped with the topology consisting of the empty set and the entire space.
  \end{description}

  The category subsets of a set and the category of open sets in a topological space are both \emph{bounded lattices}. Or, as there is either no morphism (if two sets are incomparable) or one morphism (if one set is contained in the other), we need to find objects that are either smaller or greater than all other objects. These are the empty set and the set~$X$.
\end{exercise}

\begin{exercise}
  Take~$s\in S$ a zerodisivor and let~$b\in A$ such that~$bs=0$. Now the image of~$b$ is equal to zero as
  \begin{equation}
    \frac{b}{1}=\frac{0}{1}\Longleftrightarrow s\left( b-0 \right)=0.
  \end{equation}

  Conversely, take~$a,b\in A$ and~$a\neq b$ such that their images are equal in the localization. That means there exists an~$s\in S$ such that~$s(a-b)=0$, so~$a-b\neq 0$ is a zerodivisor as~$0\notin S$.
\end{exercise}

\begin{exercise}
  \label{exercise:23d}
  The~$A$\nobreakdash-algebra~$S^{-1}A$ is a member of this category: an element~$s\in S$ is a unit as it is inverted by~$1/s$. It is furthermore initial among these algebras because the unique morphism~$\overline{\varphi}\colon S^{-1}A\to B$ is given by~$\overline{\varphi}(r/s)=\varphi(r)\varphi(s)^{-1}$ where~$\varphi\colon A\to B$ is the structure map.

  Now this morphism~$\overline{\varphi}$ is unique because if~$\psi$ would be another morphism extending~$\mathrm{i}\colon A\to S^{-1}A$ we'd find
  \begin{equation}
    \psi\left( \frac{r}{s} \right)=\psi\left( \frac{r}{1} \right)\psi\left( \frac{1}{s} \right)=\varphi(r)\varphi(s)^{-1}
  \end{equation}
  as we split the fraction into parts on which~$\mathrm{i}$ works.
\end{exercise}

\begin{exercise} % TODO this might be a little short
  The definition is already given in the hint. The checks for the~$S^{-1}A$\nobreakdash-module structure are trivial and the universal property is satisfied by the proof from~\autoref{exercise:23d}.
\end{exercise}

\begin{exercise}
  The isomorphism is given by
  \begin{equation}
    \frac{1}{s}\left( m_1,\dots,m_n \right)\mapsto \left( \frac{m_1}{s},\frac{m_2}{s},\dots,\frac{m_n}{s} \right)
  \end{equation}
  with the inverse map being
  \begin{equation}
    \left( \frac{m_1}{s_1},\ldots,\frac{m_n}{s_n} \right)\mapsto\frac{1}{\prod_{i=1}^n s_i}\left( m_1\prod_{i\neq 1}s_i,\ldots,m_n\prod_{i\neq n}s_i \right).
  \end{equation}

  In the infinite case the product of the nominators is not defined. In the scenario of the hint that is given, the image under the inverse map has both a division by zero and a multiplication by infinity.
\end{exercise}

\begin{exercise}
  It is possible to proof that
  \begin{equation}
    \mathbb{Z}/(m)\otimes_\mathbb{Z}\mathbb{Z}/(n)\cong\mathbb{Z}/(d)
  \end{equation}
  with~$d\colonequals\gcd(m,n)$. In order to do so: observe that
  \begin{equation}
    x\otimes y=\left( xy \right)\otimes 1=xy\left( 1\otimes 1 \right)
  \end{equation}
  hence~$1\otimes 1$ is the generator of a cyclic group that represents the tensor product. Now~$d(1\otimes 1)=0$ because both~$m(1\otimes 1)$ and~$n(1\otimes 1)$ are zero by bringing it into the correct factor and using B\'ezout's identity. So the order of the cyclic group divides~$d$.
  
  Now we can map the direct product into~$\mathbb{Z}/(d)$ in an obvious way, which induces a map from the tensor product to~$\mathbb{Z}/(d)$. The element~$1\otimes 1$ is mapped to~$1$ and consequently has order~$d$, so there is an element of order \emph{at least~$d$}. Hence we have obtained the desired isomorphism.

  In this special case:~$\gcd(10,12)=2$.
\end{exercise}

\begin{exercise} % TODO expand / rewrite
  \label{exercise:23h}
  That~$f\otimes\identity\colon M\otimes N\to M''\otimes N$ is still surjective is obvious.

  For the surjection of~$M'\otimes N$ onto the kernel of~$f\otimes\identity$ we have to prove that in
  \begin{equation}
    M\otimes N\to (M\otimes N)/\image((M'\to M)\otimes\identity)\to M''\otimes N,
  \end{equation}
  the last morphism being the induced morphism from~$f\otimes\identity$ is invertible. Construct the induced~$\overline{\varphi}$ from
  \begin{equation}
    \map{\varphi}{M''\otimes N}{(M\otimes N)/\image((M'\to M)\otimes\identity)}{m''\otimes n}{m\otimes n+\image((M'\to M)\otimes\identity)}
  \end{equation}
  where~$m\in f^{-1}(m'')$ by surjectivity of~$f$. Now~$\varphi$ is well defined, bilinear and therefore extends to~$\overline{\varphi}$. Now the composition~$\overline{\varphi}\circ\overline{f\otimes\identity}$ is the identity, hence the induced morphism is invertible and we have obtained exactness.
\end{exercise}

\begin{exercise}
  Unique up to unique isomorphism means that the object is not necessarily unique, but all objects satisfying the universal property are isomorphic using a unique (iso)morphism (as~$f$ is said to be so). Hence these objects have trivial automorphism groups. By the same proof as for the universal property of categorical products this holds for tensor products.
\end{exercise}

\begin{exercise} % TODO expand?
  The construction exactly quotients those objects that correspond to the bilinearity of the morphisms that are considered. Therefore all morphisms~$M\times N\to T'$ split uniquely through~$T$.
\end{exercise}

\begin{exercise} % TODO this feels very fishy
  \label{exercise:23k}
  \begin{enumerate}
    \item As stated in~\autoref{exercise:23d}: giving a ring map~$A\to B$ is the same as giving~$B$ an~$A$\nobreakdash-algebra structure. So we can consider both~$B$ and~$M$ as~$A$\nobreakdash-modules.

      We now use the fact that~$B$ is more than a module: it is an algebra. So we take by definition that~$B\otimes_A M$ interacts with scalars such that elements of~$B\setminus A$ are absorbed in the first factor, while scalar elements of~$A$ can (as by definition of the tensor product over~$A$) move around as we like.
      
      Let's tediously check all module axioms now. Take~$b_1,b_2\in B$ as scalars for the~$B$\nobreakdash-structure and~$b\otimes m,b'\otimes m'\in B\otimes_A M$ as $B$\nobreakdash-module elements. We have
      \begin{description}
        \item[$r(x+y)=rx+ry$:] This one is immediate.

        \item[$(r+s)x=rx+sx$:] We manipulate:
          \begin{equation}
            \begin{aligned}
              (b_1+b_2)(b\otimes m)&=(b_1+b_2)b\otimes m \\
              &=(b_1b+b_2)\otimes m \\
              &=b_1b\otimes m+b_2b\otimes m \\
              &=b_1(b\otimes m)+b_2(b\otimes m)
            \end{aligned}.
          \end{equation}

        \item[$(rs)x=r(sx)$:] By definition of the~$B$\nobreakdash-module structure we obtain
          \begin{equation}
            (b_1b_2)(b\otimes m)=b_1b_2b\otimes m=b_1(b_2b\otimes m).
          \end{equation}

        \item[$1_Rx=x$:] We easily obtain
          \begin{equation}
            1_B(b\otimes m)=1_Bb\otimes m=b\otimes m.
          \end{equation}
      \end{description}

      The functiorality of~$\category{Mod}_A\to\category{Mod}_B$ follows from assigning to each morphism~$f\colon M_1\to M_2$ in the first category the map~$\identity\otimes f$ in the second.

    \item Now both~$B$ and~$C$ carry the desired~$A$\nobreakdash-algebra structure necessary for this multiplication to hold and the result follows from the previous point.
  \end{enumerate}
\end{exercise}

\begin{exercise}
  The natural isomorphism is given by
  \begin{equation}
    \frac{a}{s}\otimes m\mapsto \frac{am}{s}
  \end{equation}
  which is compatible with both the~$S^{-1}A$- and~$A$\nobreakdash-module structure by definition of~$S^{-1}M$. The inverse map is obviously defined as
  \begin{equation}
    \frac{m}{s}\mapsto \frac{1}{s}\otimes m
  \end{equation}
  and this is the correct inverse because we're tensoring over the ring~$A$.
\end{exercise}

\begin{exercise}
  The condition imposed on the elements of the Cartesian product that will act as the fibered product in~$\category{Sets}$ are necessary by the commutativity of the square. The maps~$\projection_X$ and~$\projection_Y$ are the obvious projections on the first or second factor.

  A map~$W\to X\times_Z Y$ is by the agreement of compositions to~$W$ given by using the maps~$W\to X$ and~$W\to Y$ for each component. Now this must be unique: if we'd try to fit in another map the commutativity of the diagram is broken.
\end{exercise}

\begin{exercise}
  It is the \emph{intersection} of the open sets: the fibered product must be an element of the category (which in this case only contains open sets). As for morphisms: these depict inclusion ($U\subseteq V$ implies $U\to V$) and are either unique or non-existing. So if~$W$ is a map to both~$X$ and~$Y$ which map to~$Z$, we have a chain of inclusions in which we can fit~$W\subseteq X\cap Y$.
\end{exercise}

\begin{exercise}
  The definition of fibered product depends on the choice of~$f$ and~$g$. But as~$Z$ is the final object, these morphisms are unique. Now place the product in the position reserved for the fibered product in the definition. As the compositions from~$W$ to~$Z$ agree ($Z$ being final implies uniqueness of these maps, hence equality!) we have a unique isomorphism between these two products.
\end{exercise}

\begin{exercise}
  The projection~$U\to V$ is given by the definition of the fibered product in the second square. The projection~$U\to Y$ is given by composing the projections~$U\to W$ and~$W\to Y$. The compositions of the obtained projections with both~$V\to Z$ and~$X\to Z$ agree by commutativity of the diagram.

  Now take an object~$A$ with maps~$A\to V$ and~$A\to Y$ such that the compositions with~$V\to Z$ and~$Y\to Z$ agree. We wish to consider a unique map~$A\to U$. Consider the map~$A\to W$ (which exists by chasing the diagram), by definition of the first fibered product, this map is unique.
  
  Now consider~$A\to V$, we have defined~$V\to Z$ by the composition of projections. By commutativity of the diagram~$A\to W\to X$ and~$A\to V\to X$ agree so we can use the second fibered product. So just construct a unique map~$A\to U$. We have obtained a tower of fibered products.

  Essentially, it boils down to lifting~$A\to Y$ to~$A\to W$ in a unique way.
\end{exercise}

\begin{exercise}
  Let's draw a picture.
  \begin{equation}
    \begin{tikzpicture}[description/.style = {fill = white, inner sep = 2pt}, baseline=(current bounding  box.center)]
      \matrix (m) [matrix of math nodes, row sep = 3em, column sep = 3em, text height = 1.5ex, text depth = 0.25ex]
      {
        X_1\times_Y X_2 & X_2 & X_2 \\
        X_1 & Y & \\
        X_1 & & Z \\
      };
      \draw[->] (m-1-1) edge (m-1-2)
                (m-1-2) edge (m-1-3)
                (m-1-3) edge (m-3-3)
                (m-1-1) edge (m-2-1)
                (m-2-1) edge (m-3-1)
                (m-3-1) edge (m-3-3)
                (m-1-2) edge (m-2-2)
                (m-2-1) edge (m-2-2)
                (m-2-2) edge (m-3-3);
    \end{tikzpicture}.
  \end{equation}

  By definition of~$X_1\times_Y X_2$ the maps to~$Y$ through~$X_1$ and~$X_2$ agree. So the composition with~$Y\to Z$ agrees as well. We can now put the fibered product~$X_1\times_Y X_2$ in the position of~$W$ for the definition of the fibered product~$X_1\times_Z X_2$, obtaining a unique or natural morphism~$X_1\times_Y X_2\to X_1\times_Z X_2$.
\end{exercise}

\begin{exercise} % TODO do this one
  The magic diagram still feels like magic. I will come back to this exercise.
\end{exercise}

\begin{exercise}
  Let's draw another picture.
  \begin{equation}
    \begin{tikzpicture}[description/.style = {fill = white, inner sep = 2pt}, baseline=(current bounding  box.center)]
      \matrix (m) [matrix of math nodes, row sep = 3em, column sep = 3em, text height = 1.5ex, text depth = 0.25ex]
      {
        P' & & \\
        & P & N \\
        & M & \\
      };
      \path[->] (m-2-3) edge node[above] {$\nu'$} (m-1-1)
                (m-3-2) edge node[auto] {$\mu'$} (m-1-1)
                (m-2-3) edge node[auto] {$\nu$} (m-2-2)
                (m-3-2) edge node[right] {$\mu$} (m-2-2);
      \path[->, dashed] (m-2-2) edge node[description] {$\exists!$} (m-1-1);
    \end{tikzpicture}
  \end{equation}

  When~$P=M\sqcup N$ we have obvious morphisms~$\mu$ and~$\nu$ which are the canonical injections. But~$\mu'$ and~$\nu'$ are uniquely determining~$P\to P'$ as exactly the right information is stored in~$P$.
\end{exercise}

\begin{exercise}
  We check the axioms for the ring morphism~$b\mapsto b\otimes 1$:
  \begin{description}
    \item[additivity] By definition of the tensor product of two~$A$\nobreakdash-modules we obtain~$(b_1+b_2)\otimes c=b_1\otimes c+b_2\otimes c$, hence the map is additive.  
    \item[multiplicativity] By the definition of multiplication induced on~$B\otimes_A C$ in~\autoref{exercise:23k} we obtain~$(b_1b_2)\otimes 1=(b_1\otimes 1)(b_2\otimes 1)$.

    \item[preservation of unity] By definition we have~$1\mapsto 1\otimes 1$ which acts as unity for the multiplication.
  \end{description}

  We have maps~$B\to B\otimes_A C$ and~$C\to B\otimes_A C$ by the previous observations. Now these maps agree by definition of~$\otimes_A$: the images of elements of~$A$ in either~$B$ or~$C$ can be moved back and forth after the map to~$B\otimes_A C$.

  For any other ring~$A'$ equipped with maps~$f\colon B\to A'$ and~$g\colon C\to A'$ such that they agree when composed with~$A\to B$ and~$A\to C$ we have an obvious map~$B\otimes_A C\to A'$ by taking~$b\otimes c\mapsto f(b)g(c)$. As unity is preserved under ring morphisms, this makes everything commute. By the axioms of ring morphisms, this is the unique map making everything commute as this multiplicative structure of the map is imposed.
\end{exercise}

\begin{exercise}
  Consider two maps~$g_1,g_2\colon Z\to X$ and two consecutive monomorphisms~$f_1\colon X\to Y_1$,~$f_2\colon Y_1\to Y_2$ such that~$f_2\circ f_1\circ g_1=f_2\circ f_1\circ g_2$, by associativity and~$f_2$ monomorphic we obtain~$f_1\circ g_1=f_1\circ g_2$ which by~$f_1$ monomorphic yields~$g_1=g_2$.
\end{exercise}

\begin{exercise}
  \label{exercise:23v}
  If~$f\colon X\to Y$ is a monomorphism taking~$X$ as the fibered product proves its existence. The unique map~$Z\to X\times_Y X=X$ is given by~$g_1=g_2$, where~$g_1,g_2\colon Z\to X$.

  Conversely, if the fibered product exists we have that~$f\circ g_1=f\circ g_2$ as these maps must agree. Now there is a \emph{unique} map~$Z\to X\times_Y X$ so we obtain~$g_1=g_2$ as~$g_1$ and~$g_2$ both equal the composition of this unique map with the equal projections~$X\times_Y X\to X$.

  By putting~$X$ in the position for a map~$X\to X\times_Y X$ we obtain an induced isomorphism as we can use the identity map~$X\to X$ at both sides in the diagram.
\end{exercise}

\begin{exercise}
  Using the magic diagram we obtain
  \begin{equation}
    \begin{tikzpicture}[description/.style = {fill = white, inner sep = 2pt}, baseline=(current bounding  box.center)]
      \matrix (m) [matrix of math nodes, row sep = 3em, column sep = 3em, text height = 1.5ex, text depth = 0.25ex]
      {
        X_1\times_Z X_2 & & \\
        & X_1\times_Y X_2 & X_1\times_Z X_2 \\
        & Y & Y\times_Z Y\cong Y \\
      };
      \path[->] (m-1-1) edge (m-3-2)
                (m-1-1) edge (m-2-3)
                (m-2-2) edge (m-3-2)
                (m-2-2) edge (m-2-3)
                (m-3-2) edge (m-3-3)
                (m-2-3) edge (m-3-3);
      \path[->, dashed] (m-1-1) edge (m-2-2);
    \end{tikzpicture}
  \end{equation}
  where we have~$Y\times_Z Y\cong Y$ by~\autoref{exercise:23v}. The isomorphism is immediate by the uniqueness of~$X_1\times_Z X_2\to X_1\times_Y X_2$.
\end{exercise}

\begin{exercise}
  By plugging in~$C=A$ we get the candidate~$g\colonequals\mathrm{i}_A(\identity_A)$. 
\end{exercise}


\section{Limits and colimits}

\begin{exercise}
  As stated in the exercise the morphisms~$f_j$ are the obvious projection maps. The maps~$F(m)$ for all~$m\colon j\to k$ in~$\mathcal{I}$ commute with these projections by the identification~$F(m)(a_j)=a_k$.

  Now take an object~$W$ such that the~$g_i\colon W\to A_i$ commute with all the~$F(m)$. Construct~$g\colon W\to\varprojlim_{\mathcal{I}}A_i$ by using he value~$g_i(w)$ for the~$i$th position of the direct limit. By demanding commutativity this map is necessarily unique.
\end{exercise}

\begin{exercise}
  \label{exercise:24b}
  \begin{enumerate}
    \item The rational numbers~$\mathbb{Q}$ is the object that captures all of the information contained in the~$\frac{1}{n}\mathbb{Z}$. We define a morphism~$\frac{1}{n}\mathbb{Z}\to\frac{1}{m}\mathbb{Z}$ is~$n\divides m$, by~$\frac{z}{n}\mapsto\frac{(m/n)z}{m}$. The remark in the notes after~\autoref{exercise:24c} is helpful in this respect.

    \item\label{enumerate:24-b} Take subsets~$A_j$ and~$A_k$ of~$A$ such that~$m\colon A_j\hookrightarrow A_k$ is the obvious inclusion map (if it exists). The diagram becomes
      \begin{equation}
        \begin{tikzpicture}[description/.style = {fill = white, inner sep = 2pt}, baseline=(current bounding  box.center)]
          \matrix (m) [matrix of math nodes, row sep = 3em, column sep = 3em, text height = 1.5ex, text depth = 0.25ex]
          {
            & \varinjlim_{\mathcal{I}}A_i \\
            A_j & A_k \\
          };
          \path[->] (m-2-1) edge node[auto] {$f_j$} (m-1-2)
                    (m-2-2) edge node[right] {$f_k$} (m-1-2);
          \path[right hook->] (m-2-1) edge (m-2-2);
        \end{tikzpicture}
      \end{equation}
      where~$\varinjlim_{\mathcal{I}}A_i$ should capture enough but no more of the information contained in the~$A_i$ in order to make maps out of it to compatible objects unique.

      Therefore two different elements are mapped to different elements of the colimit, but they are identified for the inclusion maps. The union captures all of this information: we have the obvious inclusions~$f_i$, the~$F(m)$ are still the inclusion maps from the previous paragraph and as to maps out of~$\varinjlim_{\mathcal{I}}A_i$ to compatible objects: these are unique because the~$g_j$ define them uniquely.
  \end{enumerate}
\end{exercise}

\begin{exercise}
  \label{exercise:24c}
  Extending the answer to~\autoref{exercise:24b}\ref{enumerate:24-b} we see that the maps out of~$\varinjlim_{\mathcal{I}}A_i$ are defined uniquely by this quotient of the disjoint union.
\end{exercise}

\begin{exercise}
  First the well-definedness:
  \begin{description}
    \item[addition] the pointwise addition is compatible with the construction because the maps are all~$A$\nobreakdash-module maps, hence the identification is preserved;
    \item[scalar multiplication] the same reasoning holds.
  \end{description}

  And this construction serves as a colimit because two elements that get identified will have an equal image out of the colimit (\ie, just one choice) while the compatibility of the maps gives us the construction of the map~$\varinjlim_{\mathcal{I}}\to W$\footnote{You might have noticed it: I am not god at this kind of arguments. Please consult a good algebra textbook.}.
\end{exercise}

\begin{exercise}
  The maps~$F(m)\colon\frac{1}{s_1}A\to\frac{1}{s_2}A$ where~$s_2=s_1s'$ are defined by~$\frac{1}{s_1}a\mapsto\frac{1}{s_2}s'a$. So an element in~$S^{-1}A$ which can be regarded as a fraction appears somewhere in the direct system and stays there by the integrality. The direct limit essentially captures all the information of the localization, for which integral domains are the most intuitive case.

  In this more general case, the torsion will disappear.
\end{exercise}

\begin{exercise}
  The diagram defining the colimit is now broken into several disjoint parts by lack of the filtered condition. The construction works for these parts and they are independent, hence these are not affected by the quotient of the direct sum.
\end{exercise}


\section{Adjoints}

\begin{exercise}
  \label{exercise:25a}
  Extending the diagram given in~(2.5.0.2) we obtain
  \begin{equation}
    \begin{tikzpicture}[description/.style = {fill = white, inner sep = 2pt}, baseline=(current bounding  box.center)]
      \matrix (m) [matrix of math nodes, row sep = 3em, column sep = 3em, text height = 1.5ex, text depth = 0.25ex]
      {
        \Mor_{\mathcal{B}}\left( F(A'),B \right) & \Mor_{\mathcal{B}}\left( F(A),B \right) & \Mor_{\mathcal{B}}\left( F(A),B' \right) \\
        \Mor_{\mathcal{A}}\left( A',G(B) \right) & \Mor_{\mathcal{A}}\left( A,G(B) \right) & \Mor_{\mathcal{A}}\left( A,G(B') \right) \\
      };
      \path[->] (m-1-1) edge node[auto] {$Ff^*$} (m-1-2)
                (m-1-2) edge node[auto] {$g^*$}  (m-1-3)
                (m-2-1) edge node[auto] {$f^*$}  (m-2-2)
                (m-2-2) edge node[auto] {$Gg^*$} (m-2-3)
                (m-1-1) edge node[auto] {$\tau_{A',B}$} (m-2-1)
                (m-1-2) edge node[auto] {$\tau_{A,B}$}  (m-2-2)
                (m-1-3) edge node[auto] {$\tau_{A,B'}$} (m-2-3);
    \end{tikzpicture}
  \end{equation}
  in an obvious way, where~$g^*$ is \emph{appending} the morphism to~$f\colon F(A)\to B$.
\end{exercise}

\begin{exercise} % TODO check whether I got all the symbols correct
  The map~$\eta_A$ should behave somewhat like an identity. If we take
  \begin{equation}
    \eta_A\colonequals\tau_{A,F(A)}^{-1}\left( \identity_{F(A)} \right) 
  \end{equation}
  with~$\identity_{F(A)}\in\Mor_{\mathcal{B}}(F(A),F(A))$ we find the correct definition.
  
  Analogously we take
  \begin{equation}
    \epsilon_B\colonequals\tau_{G(B),B}^{-1}\left( \identity_{G(B)} \right)
  \end{equation}
  in~$\Mor_{\mathcal{A}}(G\circ F(A),A)$ where~$\identity_{G(B)}\in\Mor_{\mathcal{A}}(G(B),G(B))$.
\end{exercise}

\begin{exercise}
  Taking~$f\colon M\otimes_A N\to P$, transform it to~$g\colon M\to\Hom_A(N,P)$ by defining~$g(m)(-)\colonequals f(m\otimes-)$. This is a bijection:
  \begin{description}
    \item[injective] Take~$f_1,f_2$ such that~$f_1\neq f_2$, \ie, there exist~$m\in M$ and~$n\in N$ such that~$f_1(m\otimes n)\neq f_2(m\otimes n)$. Taking the corresponding~$g_i$ we obtain~$g_1(m)(-)\neq g_2(m)(-)$ as their values in the point~$n$ differ.

    \item[surjective] By the universal property of the tensor product.
  \end{description}
\end{exercise}

\begin{exercise}
  \label{exercise:25d}
  It just boils down to unwinding the diagram from~\autoref{exercise:25a}. Set~$F\colonequals-\otimes_A N$ and~$G\colonequals\Hom_A(N,-)$, $A\colonequals M$, $A'\colonequals M'$, $B\colonequals P$ and~$B'\colonequals P'$. Now taking~$h\colon M'\otimes_A N\to P$, chasing the diagram using~$\tau_{A,B}\circ Ff^*$ we obtain~$h\circ (f\otimes\identity)(-_M\otimes-_N)$ first and then~$h(f(-_M))(-_N)$.

  Chasing the diagram in the other direction, through~$f^*\circ\tau_{A',B}$ we first obtain~$h(-_M)(-_N)$ and then end up with~$h(f(-_M))(-_N)$, hence equality. The rest is analogous, now you have to compose with the map~$g\colon B\to B'$.
\end{exercise}

\begin{exercise} % TODO prove adjointness
  Let's take the equivalence relation described in the notes with \emph{pointwise addition}. Now we check the group axioms:
  \begin{description}
    \item[closure] taking~$(a,b)$ and~$(c,d)$ in~$S\times S$, the pointwise sum is~$(a+c,d+e)$ and both components are defined by the binary operation in the semigroup;
    \item[associativity] again, the pointwise sum is associative by the associativity of the binary operation in the semigroup;
    \item[identity] the element~$(s,s)$ for~$s\in S$ is the identity: all of them are identified by the equivalence relation and we have~$(a,b)\sim(a+s,b+s)$ because~$a+b+s+e=b+a+s+e$ as~$S$ is an \emph{abelian} semigroup;
    \item[inverse element] the inverse element of~$(a,b)$ is given by~$(b,a)$: their sum is~$(a+b,b+a)$ which is a representative of the equivalence class of the identity.
  \end{description}

  The map~$S\to\groupification(S)$ is given by choosing a fixed element~$e$ and mapping~$s$ to~$\overline{(s,e)}$ in~$S\times S/{\sim}$.

  The adjointness is still to come.
\end{exercise}

\begin{exercise}
  By filling in~$\pi=\identity_S$ in the universal property defining groupification we easily see the unique morphism~$G\to G'$ is given by~$S\to G$.
\end{exercise}

\begin{exercise} % TODO prove adjointness
  Because by definition~$1\in S$ we have the desired inclusion of categories~$\category{Mod}_{S^{-1}A}\hookrightarrow\category{Mod}_A$ as every~$S^{-1}$\nobreakdash-module is an~$A$\nobreakdash-module using~$a\mapsto a/1$. The converse doesn't hold obviously.

  Now this embedding is fully faithful: every~$A$\nobreakdash-module morphism~$M\to M'$ is an~$S^{-1}A$\nobreakdash-module morphism: the localization can occur either before or after the morphism by the corresponding linearity.

  The adjointness is still to come.
\end{exercise}


\section{(Co-)kernels and exact sequences (an introduction to abelian categories)}

\begin{exercise}
  By the Freyd-Mitchell embedding theorem, we can diagram-chase elements. By the element-wise definition of~$\image f^i$ there is an injection~$\image f^i\hookrightarrow A^{i+1}$ and likewise the cokernel is \emph{defined as}~$A^{i+1}/\image f^i$, hence we immediately obtain the surjection in the second part of the diagram.

  The~$i$th cohomology~$\homology^i(A^\bullet)$ is defined as~$\ker f^i/\image f^{i-1}$ which gives us the injection into~$\coker f^{i-1}$, defined as~$A^i/\image f^{i-1}$, by the injection~$\ker f^{i}\hookrightarrow A^i$. For the surjectivity of~$\coker f^{i-1}\to\image f^i$ we use the fact that it is a complex:~$f^i\circ f^{i-1}=0$.
\end{exercise}

\begin{exercise}
  We first use the fact from linear algebra that for the exact sequence
  \begin{equation}
    0\to A^1\to A^2\to A^3\to 0
  \end{equation}
  where~$\dim A^2=\dim A^1+\dim A^3$.

  Now by taking the long sequence apart and setting~$A^1\colonequals\image d^i=\ker d^{i+1}$ and likewise~$A^3\colonequals\image d^i=\ker d^{i+1}$ we can chain all these sums together and obtain the desired equality.
\end{exercise}

\begin{exercise}
  The key insight is: \emph{positionwise}. The category~$\category{Com}_{\category{Mod}_A}$ is additive by imposing the positionwise (as in: each position in a complex) structures. The kernels and cokernels of maps of complexes are defined by putting the appropriate (co)kernels at each position. The additional axioms of an abelian category follow likewise.
\end{exercise}

\begin{exercise} % TODO expand
  The truth lies in the commutativity of diagram~(2.6.4.5), but I have to come up with a good formulation.
\end{exercise}

\begin{exercise}
  This is by breaking apart the short exact sequences and putting them back together equivalent to the definition.
\end{exercise}

\begin{exercise} % TODO do this
  \begin{enumerate}
    \item 

    \item See~\autoref{exercise:23h}.

    \item 

    \item 
  \end{enumerate}
\end{exercise}
