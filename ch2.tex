\section{Motivation}

There are no exercises in this section.


\section{Categories and functors}

\begin{exercise}
  \begin{enumerate}
    \item The elements of the group(oid) correspond to the morphisms. As every morphism is an isomorphism, we can only compose morphisms on the same object. Now in case of a single object, all axioms for a group are satisfied, as isomorphisms lead to inverse elements.

    \item Take the category of a group and copy the unique object, together with all its morphisms. Voil\`a, a groupoid.

      A natural example of groupoids is the fundamental groupoid of a topological space. You cannot combine loops that have different base points.
  \end{enumerate}
\end{exercise}

\begin{exercise}
  By definition of \emph{invertible element of~$\Mor(A,A)$} the automorphisms form a group: we get composition and associativity from the category and the identity and inverse from our choice of elements.
  
  In case of~$\category{Sets}$ the automorphisms are the permutations of the set, \ie, bijections. In case of~$\category{Vec}_k$ the automorphisms are bijective linear self-maps.

  By conjugation, isomorphic objects have isomorphic automorphism groups.
\end{exercise}

\begin{exercise} % TODO I should probably do this
  Linear algebra exercise. I will do this one if I feel inspired.
\end{exercise}

\begin{exercise}
  A basis for a finite-dimensional vectorspace has a well-defined cardinality, defining its dimension. So the inverse functor~$\category{f.d.\ Vec}_k\to\mathcal{V}$ maps an~$n$\nobreakdash-dimensional vectorspace~$V$ to~$k^n$, while every linear map between finite-dimensional vectorspaces can be written against a choice of bases. As we were allowed to pick a basis for each vectorspace simultaneously, every linear map admits by linear algebra magic with tons of indices a representation as a matrix.
\end{exercise}


\section{Universal properties determine an object up to unique isomorphism}

\begin{exercise}
  \label{exercise:23a}
  Take~$A$ and~$B$ initial objects. As there exist (unique) morphisms~$A\to B$ and~$B\to A$ by definition of an initial object, we can compose them, obtaining morphisms~$A\to A$ and~$B\to B$. But the identity is another candidate for this morphisms, so by uniqueness of the morphisms~$A$ and~$B$ are isomorphic.

  The proof for final objects is completely the same.
\end{exercise}

\begin{exercise}
  \begin{description}
    \item[$\category{Sets}$] The initial object is the empty set~$\emptyset$, the singleton~$\left\{ x \right\}$ is the final object (all singletons are isomorphic as stated before in~\autoref{exercise:23a}).
    \item[$\category{Rings}$] As the image of~$1\in\mathbb{Z}$ determines the entire ring morphism, the ring of integers~$\mathbb{Z}$ is the initial object. The final object is the trivial ring (in which~$0=1$).
    \item[$\category{Top}$] The initial object is the empty set~$\emptyset$ equipped with the topology consisting of~$1$~open set, the final object is the singleton equipped with the topology consisting of the empty set and the entire space.
  \end{description}

  The category subsets of a set and the category of open sets in a topological space are both \emph{bounded lattices}. Or, as there is either no morphism (if two sets are incomparable) or one morphism (if one set is contained in the other), we need to find objects that are either smaller or greater than all other objects. These are the empty set and the set~$X$.
\end{exercise}

\begin{exercise}
  Take~$s\in S$ a zerodisivor and let~$b\in A$ such that~$bs=0$. Now the image of~$b$ is equal to zero as
  \begin{equation}
    \frac{b}{1}=\frac{0}{1}\Longleftrightarrow s\left( b-0 \right)=0.
  \end{equation}

  Conversely, take~$a,b\in A$ and~$a\neq b$ such that their images are equal in the localization. That means there exists an~$s\in S$ such that~$s(a-b)=0$, so~$a-b\neq 0$ is a zerodivisor as~$0\notin S$.
\end{exercise}

\begin{exercise}
  \label{exercise:23d}
  The~$A$\nobreakdash-algebra~$S^{-1}A$ is a member of this category: an element~$s\in S$ is a unit as it is inverted by~$1/s$. It is furthermore initial among these algebras because the unique morphism~$\overline{\varphi}\colon S^{-1}A\to B$ is given by~$\overline{\varphi}(r/s)=\varphi(r)\varphi(s)^{-1}$ where~$\varphi\colon A\to B$ is the structure map.

  Now this morphism~$\overline{\varphi}$ is unique because if~$\psi$ would be another morphism extending~$\mathrm{i}\colon A\to S^{-1}A$ we'd find
  \begin{equation}
    \psi\left( \frac{r}{s} \right)=\psi\left( \frac{r}{1} \right)\psi\left( \frac{1}{s} \right)=\varphi(r)\varphi(s)^{-1}
  \end{equation}
  as we split the fraction into parts on which~$\mathrm{i}$ works.
\end{exercise}

\begin{exercise} % TODO this might be a little short
  The definition is already given in the hint. The checks for the~$S^{-1}A$\nobreakdash-module structure are trivial and the universal property is satisfied by the proof from~\autoref{exercise:23d}.
\end{exercise}

\begin{exercise}
  The isomorphism is given by
  \begin{equation}
    \frac{1}{s}\left( m_1,\dots,m_n \right)\mapsto \left( \frac{m_1}{s},\frac{m_2}{s},\dots,\frac{m_n}{s} \right)
  \end{equation}
  with the inverse map being
  \begin{equation}
    \left( \frac{m_1}{s_1},\ldots,\frac{m_n}{s_n} \right)\mapsto\frac{1}{\prod_{i=1}^n s_i}\left( m_1\prod_{i\neq 1}s_i,\ldots,m_n\prod_{i\neq n}s_i \right).
  \end{equation}

  In the infinite case the product of the nominators is not defined. In the scenario of the hint that is given, the image under the inverse map has both a division by zero and a multiplication by infinity.
\end{exercise}

\begin{exercise}
  It is possible to proof that
  \begin{equation}
    \mathbb{Z}/(m)\otimes_\mathbb{Z}\mathbb{Z}/(n)\cong\mathbb{Z}/(d)
  \end{equation}
  with~$d\colonequals\gcd(m,n)$. In order to do so: observe that
  \begin{equation}
    x\otimes y=\left( xy \right)\otimes 1=xy\left( 1\otimes 1 \right)
  \end{equation}
  hence~$1\otimes 1$ is the generator of a cyclic group that represents the tensor product. Now~$d(1\otimes 1)=0$ because both~$m(1\otimes 1)$ and~$n(1\otimes 1)$ are zero by bringing it into the correct factor and using B\'ezout's identity. So the order of the cyclic group divides~$d$.
  
  Now we can map the direct product into~$\mathbb{Z}/(d)$ in an obvious way, which induces a map from the tensor product to~$\mathbb{Z}/(d)$. The element~$1\otimes 1$ is mapped to~$1$ and consequently has order~$d$, so there is an element of order \emph{at least~$d$}. Hence we have obtained the desired isomorphism.

  In this special case:~$\gcd(10,12)=2$.
\end{exercise}

\begin{exercise} % TODO expand / rewrite
  That~$f\otimes\identity\colon M\otimes N\to M''\otimes N$ is still surjective is obvious.

  For the surjection of~$M'\otimes N$ onto the kernel of~$f\otimes\identity$ we have to prove that in
  \begin{equation}
    M\otimes N\to (M\otimes N)/\Image((M'\to M)\otimes\identity)\to M''\otimes N,
  \end{equation}
  the last morphism being the induced morphism from~$f\otimes\identity$ is invertible. Construct the induced~$\overline{\varphi}$ from
  \begin{equation}
    \map{\varphi}{M''\otimes N}{(M\otimes N)/\Image((M'\to M)\otimes\identity)}{m''\otimes n}{m\otimes n+\Image((M'\to M)\otimes\identity)}
  \end{equation}
  where~$m\in f^{-1}(m'')$ by surjectivity of~$f$. Now~$\varphi$ is well defined, bilinear and therefore extends to~$\overline{\varphi}$. Now the composition~$\overline{\varphi}\circ\overline{f\otimes\identity}$ is the identity, hence the induced morphism is invertible and we have obtained exactness.
\end{exercise}

\begin{exercise}
  Unique up to unique isomorphism means that the object is not necessarily unique, but all objects satisfying the universal property are isomorphic using a unique (iso)morphism (as~$f$ is said to be so). Hence these objects have trivial automorphism groups. By the same proof as for the universal property of categorical products this holds for tensor products.
\end{exercise}

\begin{exercise} % TODO expand?
  The construction exactly quotients those objects that correspond to the bilinearity of the morphisms that are considered. Therefore all morphisms~$M\times N\to T'$ split uniquely through~$T$.
\end{exercise}

\begin{exercise} % TODO this feels very fishy
  \begin{enumerate}
    \item As stated in~\autoref{exercise:23d}: giving a ring map~$A\to B$ is the same as giving~$B$ an~$A$\nobreakdash-algebra structure. So we can consider both~$B$ and~$M$ as~$A$\nobreakdash-modules.

      We now use the fact that~$B$ is more than a module: it is an algebra. So we take by definition that~$B\otimes_A M$ interacts with scalars such that elements of~$B\setminus A$ are absorbed in the first factor, while scalar elements of~$A$ can (as by definition of the tensor product over~$A$) move around as we like.
      
      Let's tediously check all module axioms now. Take~$b_1,b_2\in B$ as scalars for the~$B$\nobreakdash-structure and~$b\otimes m,b'\otimes m'\in B\otimes_A M$ as $B$\nobreakdash-module elements. We have
      \begin{description}
        \item[$r(x+y)=rx+ry$:] This one is immediate.

        \item[$(r+s)x=rx+sx$:] We manipulate:
          \begin{equation}
            \begin{aligned}
              (b_1+b_2)(b\otimes m)&=(b_1+b_2)b\otimes m \\
              &=(b_1b+b_2)\otimes m \\
              &=b_1b\otimes m+b_2b\otimes m \\
              &=b_1(b\otimes m)+b_2(b\otimes m)
            \end{aligned}.
          \end{equation}

        \item[$(rs)x=r(sx)$:] By definition of the~$B$\nobreakdash-module structure we obtain
          \begin{equation}
            (b_1b_2)(b\otimes m)=b_1b_2b\otimes m=b_1(b_2b\otimes m).
          \end{equation}

        \item[$1_Rx=x$:] We easily obtain
          \begin{equation}
            1_B(b\otimes m)=1_Bb\otimes m=b\otimes m.
          \end{equation}
      \end{description}

      The functiorality of~$\category{Mod}_A\to\category{Mod}_B$ follows from assigning to each morphism~$f\colon M_1\to M_2$ in the first category the map~$\identity\otimes f$ in the second.

    \item Now both~$B$ and~$C$ carry the desired~$A$\nobreakdash-algebra structure necessary for this multiplication to hold and the result follows from the previous point.
  \end{enumerate}
\end{exercise}

\begin{exercise}
  The natural isomorphism is given by
  \begin{equation}
    \frac{a}{s}\otimes m\mapsto \frac{am}{s}
  \end{equation}
  which is compatible with both the~$S^{-1}A$- and~$A$\nobreakdash-module structure by definition of~$S^{-1}M$. The inverse map is obviously defined as
  \begin{equation}
    \frac{m}{s}\mapsto \frac{1}{s}\otimes m
  \end{equation}
  and this is the correct inverse because we're tensoring over the ring~$A$.
\end{exercise}

\begin{exercise}
  The condition imposed on the elements of the Cartesian product that will act as the fibered product in~$\category{Sets}$ are necessary by the commutativity of the square. The maps~$\projection_X$ and~$\projection_Y$ are the obvious projections on the first or second factor.

  A map~$W\to X\times_Z Y$ is by the agreement of compositions to~$W$ given by using the maps~$W\to X$ and~$W\to Y$ for each component. Now this must be unique: if we'd try to fit in another map the commutativity of the diagram is broken.
\end{exercise}

\begin{exercise}
  It is the \emph{union} of the open sets: the fibered product must be an element of the category (which in this case only contains open sets). As for morphisms: these depict inclusion and are either unique or non-existing. So if~$W$ is a map to both~$X$ and~$Y$ which map to~$Z$, we have a chain of inclusions in which we can fit~$X\cup Y$.
\end{exercise}

\begin{exercise}
  The definition of fibered product depends on the choice of~$f$ and~$g$. But as~$Z$ is the final object, these morphisms are unique. Now place the product in the position reserved for the fibered product in the definition. As the compositions from~$W$ to~$Z$ agree ($Z$ being final implies uniqueness of these maps, hence equality!) we have a unique isomorphism between these two products.
\end{exercise}
